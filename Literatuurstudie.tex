%==============================================================================
% Sjabloon onderzoeksvoorstel bachproef
%==============================================================================
% Gebaseerd op document class `hogent-article'
% zie <https://github.com/HoGentTIN/latex-hogent-article>

% Voor een voorstel in het Engels: voeg de documentclass-optie [english] toe.
% Let op: kan enkel na toestemming van de bachelorproefcoördinator!
\documentclass{hogent-article}
\addbibresource{literatuur.bib}


\studyprogramme{Professionele bachelor toegepaste informatica}
\course{Bachelorproef}
\assignmenttype{Literatuurstudie bedrijfscasus}

\academicyear{2022-2023} 

% TODO: Werktitel
\title{De rol van de business-analist in agile projecten}

% TODO: Studentnaam en emailadres invullen
\author{Arnout Houtave}
\email{Arnout.Houtave@student.hogent.be}
% TODO: Medestudent
\author{Dylan Pylser}
\email{dylan.pylyser@student.hogent.be}
\author{Lukas Van Dorpe}
\email{lukas.vandorpe@student.hogent.be}
\author{Laurens Arnauts}
\email{laurens.arnauts@student.hogent.be}


\specialisation{Functional \& Business Analysis}
\keywords{agile projecten, business-analist, bedrijfscase}

\begin{document}

\begin{abstract}
 Still to do.
\end{abstract}

\tableofcontents


\section{Opdrachtomschrijving}
Het agile manifesto stelt "Working software over comprehensive documentation". Betekent dit dat lastenboeken overbodig zijn? Wat is dan nog de taak van de business-analist? Is business-analyse in agile projecten nog altijd een rol of enkel een skill? Wordt in zo'n project de rol van de opdrachtgever ("de business") minder relevant of juist belangrijker?  
\\~\\
Het zijn dan ook deze vragen waarover wij als studenten een mening dienen te vormen aan de hand van kritisch onderzoek. Dit onderzoek bestaat uit twee opeenvolgende luiken. Allereerst dienen wij een literatuurstudie uit te voeren, die we gebruiken om onze eigen hypothese over bovenstaande vragen vorm te geven. Onze hypothese in het achterhoofd houdend, zullen wij deze vervolgens toetsen bij ten minste twee bedrijven die reeds met ons onderwerp in het werkveld geconfronteerd werden. Eens dat wij alle informatie verzameld hebben, dienen wij deze daarna af te toetsen aan onze eigen hypothese en te kijken in hoeverre deze er mee overeenkomt of verschilt.
\newpage

\section{Literatuurstudie}
\subsection{Inleiding}

Het agile manifesto stelt:

\begin{itemize}
  \item Mensen en hun onderlinge interactie boven processen en hulpmiddelen
  \item Werkende software boven allesomvattende documentatie
  \item Samenwerking met de klant boven contractonderhandelingen
  \item Inspelen op verandering boven het volgen van een plan
\end{itemize}
\autocite{fowler2001agile}
\\~~\\
Aangezien nergens in een agile omgeving, of in het agile manifesto, expliciet de taken van een business analist beschreven staan, zou men aanvankelijk kunnen denken dat de rol van business analist zou mogen geschrapt worden binnen agile projecten. Het is echter niet omdat de rol van de business analist weinig of niet ter sprake komt wanneer men het over agile heeft, dat er niet gedaan wordt aan business analyse tijdens agile projecten. 
\\~~\\
Er wordt immers verwacht dat, dankzij agile's focus op het voorzien van waarde voor de klant, ieder lid van het team zich op regelmatige basis en collaboratief bezig houdt met business analyse. Hiernaast dient er ook aandacht besteed te worden aan een bijkomende verandering die agile werken teweeg brengt: hoewel de rollen in agile veelal een algemenere vorm aannemen dan wanneer deze gebruikt worden in een traditionelere omgeving, kan men nog steeds spreken over rollen. 
\\~~\\
Één van deze rollen, de \emph{product owner}, vormt de ultieme beslisser en vertegenwoordiger wanneer het gaat over de noden van de business tijdens het project. Deze persoon legt onder andere de productvisie vast, bepaalt welke requirements prioriteit krijgen en is verantwoordelijk om de noden van de stakeholders te begrijpen en te vertegenwoordigen. Hierdoor is het dus vanzelfsprekend dat de persoon die deze rol toegewezen wordt, vertrouwd is met tal van vaardigheden die eigen zijn aan business analyse. In deze rol blinkt de business analist dan ook uit, zoniet als product owner dan wel als iemand die de eigenlijke product owner steeds ondersteunt. 
\\~~\\
Naast de functie van business advisor zou de business analist bovendien ook dienst kunnen doen als business coach, waarbij hij zijn skills en know-how kan inzetten als de analyse specialist van het team. Dit houdt onder andere in dat hij de samenwerking tussen het development team en de stakeholders vlot doet verlopen en voorbeelden genereert die gebruikt kunnen worden om duidelijk de wensen van de business te schetsen.\autocite{mcdonald2017does}
\\~~\\
Binnen agile heeft ten slotte ook een natuurlijke evolutie plaatsgevonden naar multidisciplinaire teams die bestaan uit een optimale balans tussen technische kennis en softskills. De ultieme combinatie van deze twee categoriën van skills wordt dan ook op verschillende beschreven en gemodeleerd. bijvoorbeeld T-shaped, Pi-shaped en multi-shaped. Wanneer iemand T-shaped is betekent dat deze persoon een combinatie van diverse skills, alsook dieptekennis van één vakgebied, in zijn repertoire heeft. Pi-shaped betekent dan weer dat iemand twee vakgebieden heeft waar hij zich in heeft verdiept, in combinatie met de eerder vernoemde breedere skills. Ten slotte slaat multi-shaped op eender welke andere combinatie van skills en kennis.\autocite{heijnerelevante}

\subsection{De hypothese}

Doordat agile draait om werken met multidisciplinaire teams, bestaande uit leden die een optimale combinatie van soft and technische skills met zich meebrengen, stelt onze hypothese dat de skills en kennis van een business analist een cruciale rol speelt en ook zal blijven spelen binnen de agile methodologie. Hierbij zal het echter wel belangrijk zijn dat de persoon die de taken van de business analist op zich neemt, zich ook zal verdiepen in zaken die buiten deze rol liggen.

\newpage

\section{interviews}
\subsection{Algemene Vragen}

\begin{enumerate}
  \item Zoudt u zich even kunnen voorstellen?
  \item Bent u binnen uw bedrijf een pure business analist of neemt u ook nog andere rollen op?
  \item Wat zijn uw unieke taken als business analist binnen het bedrijfsproces?
  \item Vindt u dat de rol van business analist binnen Agile projecten eerder een rol of een skill is?
  \item Welke pijnpunten ervaart u met Agile werken als business analist?
  \item Wat zijn volgens u de belangrijkste kwaliteiten die een business analist in Agile projecten moet over beschikken?
  \item Het Agile manifesto stelt `werkende software boven gedetailleerde documentatie`. Loopt deze stelling gelijk met uw ervaring in het werkveld?
  \item Hoe ziet volgens u de rol van business analist er in de toekomst uit?
\end{enumerate}

\subsection{ Samenvatting interview Baloise}

Daan Robberecht is business analist bij het Zwitserse holdingbedrijf Baloise. Daan had geen achtergrond in de IT of analyse en is via een bepaald pad toch binnen de business analyse binnengerold. Door zijn interesse voor projecten wou hij graag die kant opstromen, hij heeft nog geen spijt gehad van deze beslissing.
\\~~\\
Meneer Robberecht is allesbehalve het theoretische voorbeeld van een business analist. Naast zijn scope die vooral gefocust is op de business, neemt hij ook de technische/functionele analyse en grote delen van integratie en business testing op zich. Deze rollen moest hij opnemen wegens het ziekvallen van een van de projectanalisten. Daarnaast regelt hij ook binnen het project de opleidingen voor projectleden, vragen in verband met de business en de toegangsrechten voor projectleden. Deze vallen ook deels buiten de scope van een “pure” business analist. Daartegenover zijn er ook vele taken die hij binnen zijn functie als business analist moet vervullen. Hij nam als voorbeeld het huidige project waarin hij actief bezig is omtrent brandschade. Als analist moet hij een balans vinden tussen de klantzijde en de businesszijde. Als het project bijvoorbeeld over automatisatie gaat, dan moet hij de impact kunnen schetsen die zowel de klant als de business kan ondervinden van deze automatisatie. Dit gaat dan zowel over de systemen, de interne gebruikers, de externe gebruikers en stakeholders.
\\~~\\
Volgens meneer Robberecht is business analyse een skill. Als argument gaf hij dat veel analisten binnen Baloise al meer dan een rol op zich moeten nemen. Hij vindt dat een business analist “T-Shaped” moet zijn. Met deze term wilt hij zeggen dat een analist in veel dingen moet goed zijn en zeker een expert moet zijn in een of meerdere domeinen. Bovendien moet een analist durven over de “muur” te kijken, nieuwsgierig zijn naar de andere zaken. Als een business analist zich alleen maar blijft focussen op de business, dan gaat de kwaliteit van de analyse achteruit. Het nieuwsgierige aspect kan daarbij ook helpen. Hoe meer je weet van anderen (bv. developers), hoe beter de analyse zal worden omdat je een beter algemeen beeld hebt van de zaken. Waardoor de samenhang in het project beter is en zo een kwaliteitsvollere oplevering kunt behalen.
\\~~\\
Sinds vorig jaar is Baloise afgestapt van de Waterfall methode en zijn ze overgestapt naar de Agile methodologie. Tijdens onze opleiding wordt meestal Agile boven Waterfall verkozen, bij Baloise hebben ze lang gewacht voor deze overstap. Meneer Robberecht heeft de vele pijnpunten van Waterfall ondervonden, maar merkt dat de Agile manier van werken ook nog steeds moet wennen. Het Agile Manifesto wordt hier en daar wel toegepast binnen de projecten. Voor de meeste projectleden is de werkende software belangrijker dan de documentatie en durven ze de documentatie over te slaan wanneer de tijd dringt. Daarnaast is het niet zo eenvoudig om de Agile ceremonies zoals Daily Stand-up, Sprint Review, Retrospective… uit te voeren wanneer ze in tijdsnood komen. Volgens meneer Robberecht hebben ze eerder de aspecten die ze goed vonden van Agile overgenomen (bv. sprints), maar hij merkt ook dat dit ook niet de bedoeling is en dit kan voor problemen zorgen in verband met de planning, documentatie of onderlinge communicatie.
\\~~\\
De toekomst van de business analist ziet er dan ook vrij rooskleurig uit volgens meneer Robberecht. Hij vindt dat de periode rond COVID-19 de digitalisering bij bedrijven meer in gang heeft gezet, en dat daardoor de vraag zal stijgen naar business analisten en analisten in het algemeen. Hij deelde met ons ook zijn redenering in verband met bedrijven. Hij vertelde dat er veel bedrijven vanuit de buitenkant heel meegaand zijn met de tijd en nieuwste technologie, maar dat er binnenin nog veel principes en methodologieën worden toegepast die sterk verouderd zijn. Hierdoor zullen zij ook de overstap moeten maken naar de nieuwste technologieën en methodes, hierdoor zal de nood aan analisten met deze kennis sterk stijgen. Vervolgens denkt hij dat business analisten meer “T-shaped” zullen worden en dat de functie omtrent Business Analyse eerder zal veralgemeend worden en veel meer zal inhouden dan alleen de businesszijde. Hij vindt dus weldegelijk dat business analisten nodig zijn binnen Agile projecten.


\subsection{Vergelijking met de theorie }

Meneer Robberecht raakte een aantal punten tijdens het interview aan die aansluiten met hetgeen wat wij zelf zagen terugkomen in onze literatuurstudie. Ten eerste is dat de skills en kennis van de business analist zeker en vast hun plaats hebben binnen de agile methodologie, zowel nu als in de toekomst. Het is wél zo dat de rol van pure business analist echter niet meer echt bestaat aangezien deze persoon verwacht wordt om heel wat andere deeltaken op zich te nemen. Net zoals in de theorie, vertelde hij dat een business analist T-shaped dient te zijn in een agile environment. Bovendien moet hij ook nieuwsgierig zijn en over de muur van de vakgebieden durven kijken.

Aan de andere kant waren er ook zaken waar wij nog niet echt hadden bij stilgestaan tijdens onze lessen op school in voorgaande jaren. Zo leek de switch van de waterval naar de agile methode voor ons een klare en eenvoudige zaak, wat in de praktijk helemaal niet zo bleek te zijn. Zo stuitte meneer Robberecht niet alleen zelf op allerlei moeilijkheden wanneer hij de switch naar Agile diende te maken, maar was er ook geen sprake van een volledige switch naar agile op de werkvloer. De werknemers van Baloise kozen er voor om enkel de goede aspecten van Agile over te nemen en de zaken die voor moeilijkheden dreigden te zorgen weg te laten.

\subsection{ Samenvatting interview PHPro}

Paulien De Pauw is een Business Analist en Product Owner (PO) bij het Belgische bedrijf PHPro. Mevrouw De Pauw heeft aan de Hogeschool Gent de bachelor Toegepaste Informatica behaald en heeft daarna nog een Ba-na-ba  (Bachelor na bachelor) Advanced Business Management met afstudeerrichting Data Analyse behaald aan de Leuvense UCLL. Ze is nu reeds een anderhalf jaar aan de slag bij PHPro en is gebeten door haar vakgebied. 

Binnen PHPro zijn de meeste rollen omtrent analyse tweedelig, je bent bijvoorbeeld zowel een Business Analist als een Product Owner in het geval van de interviewee. Mevrouw De Pauw is daardoor de brug tussen de klant en het projectteam. Haar technische aanleg wordt gebruikt om de verwachtingen van de klant technisch om te zetten naar het projectteam, bijvoorbeeld aan de hand van haar analyses. Daarnaast is ze als PO verantwoordelijk voor de planning, het klaarmaken van de sprints en het helpen van de Developers indien ze vastzitten of onduidelijkheden hebben. Daarnaast vertegenwoordigd ze dan ook de business en weet ze wat er langs de bedrijfskant moet gebeuren en mogelijk is.

Mevrouw De Pauw wist ons te vertellen dat deze aanpak van tweedelige rollen wel bedrijfsspecifiek is. Ze ondervindt heel veel voordelen bij haar tweedelige functie, de vertaling van de klant naar iets technisch aan de hand van analyses, én het beheren en prioriteren van de backlog gaan heel goed met elkaar samen. Ze vindt persoonlijk dat de stelling “elke business analist prioriteert de backlog” niet echt vanzelfsprekend is en ook voor moeilijkheden zorgt indien dit wordt toegepast in een bedrijf. Mevrouw De Pauw vond dat het afzonderen van een Business Analist en een Product Owner tot aparte rollen voor een extra barrière zorgt. Een analist kent zijn analyses door en door, indien een Developer dan met een vraag zit dan zou deze vraag worden gesteld aan de Product Owner in plaats van de analist. De PO stelt dan op zijn beurt de vraag (van de developer) aan de analist. Dit zorgt voor een “lost in translation” en dat kan vermeden worden door die twee rollen samen te nemen. Hierdoor ziet ze het afzonderen van deze rollen persoonlijk als een nadeel.

Volgens mevrouw De Pauw is business analyse een rol, dit is tegenstrijdig met wat meneer Robberecht zei tijdens het eerste interview. Zij vond dat business analyse een fulltime job kan zijn en dat analysis uitwerken een grote doorlooptijd heeft. Ze vond dan ook indien analyse als een skill wordt opgenomen, bijvoorbeeld bij een developer die naast het programmeren ook analyses uitwerkt, dat ze zullen merken dat de analyses dan van een mindere kwaliteit zijn. Natuurlijk is het ook mogelijk dat zijzelf fouten maakt als analist. De sprinten fout inschatten en inplannen zijn mogelijk binnen een project, daarvoor voorziet zij een extra houvast voor tijdens de sprints. Mevrouw De Pauw zorgt ervoor dat er buffers worden voorzien om de kans op een gefaalde sprint te verkleinen. Deze buffers zijn eigenlijk extra momenten die worden ingepland binnen een sprint die dienen als reserve, bijvoorbeeld in een sprint van tien dagen, wordt er een “bufferdag” voorzien als reserve.

Mevrouw De Pauw heeft een technische achtergrond waardoor het Agile Manifesto gekend is bij haar. PHPro volgt het Agile Manifesto niet en is niet akkoord met de stelling dat werkende software boven documentatie staat. Mevrouw De Pauw hecht veel belang aan documentatie omdat ze vindt dat dit voor rust en duidelijkheid kan zorgen voor zowel het team als de klant. Ze heeft dan zelf ook in een situatie gezeten waarbij zij midden in een project moest wisselen naar een ander project waarbij er geen documentatie beschikbaar was. Dit leidde tot veel onduidelijkheden en een minder goede analyse omdat ze geen kennis had van het nieuwe project dat al reeds bezig was. Daarnaast is PHPro ook niet 100\% verbonden aan het Agile principe. Ze maken gebruik van een hybride vorm waardoor de kenmerken van het Agile Manifesto ook niet volledig worden toegepast. In de hybride vorm nemen ze de beste kenmerken uit zowel Agile, Waterfall of eventueel andere projectvormen. Steeds meer bedrijven maken gebruik van deze vorm zodat ze zelf het beste kiezen naar hun noden.

Mevrouw De Pauw is er van overtuigd dat de rol van business analisten niet zullen verdwijnen. Ze denkt dat dit juist eerder een flexibele rol wordt zoals ze nu bij PHPro toepassen. De combinatie van Business Analysis als Product Ownership zal dus vaker voorkomen, maar ze vindt dat dit heel sectorspecifiek zal zijn. Langs de andere kant denkt ze dat eventueel het analyse gedeelte ook meer en meer door developers zullen moeten opgenomen worden, alleen denkt ze dat dit geen goede combinatie is omdat ze dan heel veel met klant moeten samenwerken. Dit zal dus zijn voor- en nadelen hebben indien deze methode wordt toegepast.




\subsection{Vergelijking met de theorie}

Ook bij PHPro wordt er tijdens het bedrijfsproces géén afzonderlijke functie uitgeschreven voor de business analist. De werknemer die deze taken binnen het projectmatig werk om zich neemt zal naast de taken van de business analist, zich ook bezig houden met andere taken en verantwoordelijkheden. Zo is mevrouw De Pauw niet enkel business analist, maar ook een technisch aangelegde informaticus. Men hier dus spreken van een Pi-shaped profiel. Net zo als de andere bedrijven die wij geïnterviewd hebben, werkt PHPro bovendien niet strikt volgens het Agile manifesto en verkiezen zij eerder een hybride vorm tijdens hun bedrijfsproces. Ook gaf zij zelf haar voorkeur aan voor documentatie, iets wat haak staat op het manifesto, en vertelde hoe het gebrek hieraan het project kan kwetsen.  

In tegenstelling tot het eerste interview met meneer Robberecht en het feit dat zij meerdere taken vervulde, is het echter opmerkelijk dat zij business analyse desalniettemin eerder zag als een rol en niet een skill. Haar argument hiervoor was dat als een developer zich zou bezighouden met zowel programmeren als analyse, dat één van deze zaken hieronder zou lijden.

Ten slotte was er ook een duidelijke overeenkomst met de theorie in verband met de toekomst van de business analist. Zo zullen de functies van de business analist vast en zeker blijven bestaan, maar zal de persoon die deze rol uitvoert zich verdiepen in meer dan dit ene vakgebied.

\subsection{ Samenvatting interview Meteor}

Glenn Jans is een Business Analist en Product Owner bij een vertakking van Cronos-group genaamd Meteor. Meteor is een zeer jong bedrijf dat gespecialiseerd is in Shopware en zich bevindt in de Digital subtak van Cronos. Meneer Jans heeft een achtergrond in de project management maar heeft toch de overschakeling naar Business Analysis gemaakt. 

Meteor werkt niet met pure business analisten. Binnen het bedrijf heb je een dubbelrol en focus je zowel op de analyse als op het product ownership en het testen van de opleveringen. Zie het als een hybride oplossing dat ook in de lijn valt van PHPro. Het bedrijf is ontstaan uit een vorig bedrijf, maar dan met nieuwe inbreng hebben ze Meteor samengesteld. Het bedrijf heeft een agile manier van werken waarbij ze absoluut Waterfall-methodes willen vermijden. Dit heeft meneer Jans zelf ook ondervonden na een slechte ervaring met Waterfall bij het vorige bedrijf. Hij vindt dat de waterval methode een langdurig proces is waarbij de kans op succes heel klein kan zijn. 

Meneer Jans wist ons te vertellen dat het bedrijf niet tot op de puntjes de agile- of scrum-methodologie toepast. Daarbij passen ze ook niet volledig het Agile Manifesto toe. Hij vindt dat het Manifesto niet altijd kan werken in het bedrijfsleven en dat het toepassen van Agile tot op de puntjes kan leiden tot bijvoorbeeld over het afgesproken budget gaan. Zeker als je regelmatig met je klant samenzit die steeds meer en meer opties en toepassingen wilt tijdens de contactmomenten, daardoor neem je een stapje terug van dat Agile principe. Daardoor nemen ze het beste van de methodologieën en passen ze deze toe wanneer het nodig is. Zeker in verband met budgetten en communicatie naar de klant hebben ze hier hun eigen werking in.

Daarnaast is meneer Jans er niet van overtuigd dat een pure business analist voor veel voordelen gaat zorgen binnen agile projecten. Hij vindt dat een pure business analist ervoor zal zorgen dat effectief zijn taken vervuld zijn maar dat de analist gewoon zijn deel “over de muur gooit” en dat de developers maar het moeten uitklaren. Daardoor vindt hij een dubbele of hybride rol een goed idee om misverstanden te vermijden en het volledig potentieel uit deze rollen te halen en te combineren in een. Hij zegt wel dat voor het analyse-gedeelte veel ervaring en kennis nodig is, voor minder ervaren analisten is het misschien een beter idee om zich dan te focussen op het PO-gedeelte van het project (zeker bij e-commerce). Kortom, naar de toekomst toe zal misschien het hybride model de toekomst zijn van de business analisten binnen agile projecten.


\subsection{Vergelijking met de theorie}

In dit interview kwam opnieuw naarvoor hoe de rol van de business analist geen zorgvoldig afgelijnde functie is binnen Agile werken. De werknemers die de verantwoordelijkheden van een business analist op zich nemen, zullen dit veelal doen in combinatie met andere functies. Zo neemt meneer Jans zowel de analyse, als de product ownership en testing van het project op zich.  Men kan dus zeggen dat, net zo als in de theorie, meneer Jans Pi-shaped dient te zijn binnen de agile omgeving van zijn bedrijf. Bovendien maakte meneer Jans ook zelf deze opmerking en vertelde hij dat een een zogenaamde dubbel- of hybride rol een echte meerwaarde vormt voor het project aangezien dat het analyse gedeelte enorm geniet van kennis te hebben van meerdere gebieden dan louter de business kant. Hier ligt dan ook volgens hem de toekomst van de business analist, wat bovendien overeenkomt met hetgeen de vakliteratuur verteld. 

Het is het ook opmerkelijk dat, net zo als in de andere interviews, het bedrijf niet volledig toegewijd is aan de Agile methodologie. Zij verkiezen dan opnieuw om enkel de beste delen van Agile toe te passen en de zaken die zij problematisch vonden om mee te werken, achterwegen te laten. 


\section{Reflectie}


\printbibliography[heading=bibintoc]

\end{document}