%==============================================================================
% Sjabloon onderzoeksvoorstel bachproef
%==============================================================================
% Gebaseerd op document class `hogent-article'
% zie <https://github.com/HoGentTIN/latex-hogent-article>

% Voor een voorstel in het Engels: voeg de documentclass-optie [english] toe.
% Let op: kan enkel na toestemming van de bachelorproefcoördinator!
\documentclass{hogent-article}
\addbibresource{literatuur.bib}


\studyprogramme{Professionele bachelor toegepaste informatica}
\course{Bachelorproef}
\assignmenttype{Literatuurstudie bedrijfscasus}

\academicyear{2022-2023} 

% TODO: Werktitel
\title{De rol van de business-analist in agile projecten}

% TODO: Studentnaam en emailadres invullen
\author{Arnout Houtave}
\email{Arnout.Houtave@student.hogent.be}
% TODO: Medestudent
\author{Dylan Pylser}
\email{dylan.pylyser@student.hogent.be}
\author{Lukas Van Dorpe}
\email{lukas.vandorpe@student.hogent.be}
\author{Laurens Arnauts}
\email{laurens.arnauts@student.hogent.be}


\specialisation{Functional \& Business Analysis}
\keywords{agile projecten, business-analist, bedrijfscase}

\begin{document}

\begin{abstract}
 Still to do.
\end{abstract}

\tableofcontents


\section{Opdrachtomschrijving}

Het agile manifesto stelt "Working software over comprehensive documentation". Betekent dit dat lastenboeken overbodig zijn? Wat is dan nog de taak van de business-analist? Is business-analyse in agile projecten nog altijd een rol of enkel een skill? Wordt in zo'n project de rol van de opdrachtgever ("de business") minder relevant of juist belangrijker?  

Het zijn dan ook deze vragen waarover wij als studenten een mening dienen te vormen aan de hand van kritisch onderzoek. Dit onderzoek bestaat uit twee opeenvolgende luiken. Allereerst dienen wij een literatuurstudie uit te voeren, die we gebruiken om onze eigen hypothese over bovenstaande vragen vorm te geven. Onze hypothese in het achterhoofd houdend, zullen wij deze vervolgens toetsen bij ten minste twee bedrijven die reeds met ons onderwerp in het werkveld geconfronteerd werden. Eens dat wij alle informatie verzameld hebben, dienen wij deze daarna af te toetsen aan onze eigen hypothese en te kijken in hoeverre deze er mee overeenkomt of verschilt.


\section{Literatuurstudie}

\subsection{Inleiding}

Het agile manifesto stelt:

\begin{itemize}
  \item Mensen en hun onderlinge interactie boven processen en hulpmiddelen
  \item Werkende software boven allesomvattende documentatie
  \item Samenwerking met de klant boven contractonderhandelingen
  \item Inspelen op verandering boven het volgen van een plan
\end{itemize}
\autocite{fowler2001agile}

Aangezien nergens in een agile omgeving, of in het agile manifesto, expliciet de taken van een business analist beschreven staan, zou men aanvankelijk kunnen denken dat de rol van business analist zou mogen geschrapt worden binnen agile projecten.

Het is echter niet omdat de rol van de business analist weinig of niet ter sprake komt wanneer men het over agile heeft, dat er niet gedaan wordt aan business analyse tijdens agile projecten. Er wordt immers verwacht dat, dankzij agile's focus op het voorzien van waarde voor de klant, ieder lid van het team zich op regelmatige basis en collaboratief bezig houdt met business analyse.Hiernaast dient er ook aandacht besteed te worden aan een bijkomende verandering die agile werken teweeg brengt: hoewel de rollen in agile veelal een algemenere vorm aannemen dan wanneer deze gebruikt worden in een traditionelere omgeving, kan men nog steeds spreken over rollen. Één van deze rollen, de \emph{product owner}, vormt de ultieme beslisser en vertegenwoordiger wanneer het gaat over de noden van de business tijdens het project. Deze persoon legt onder andere de productvisie vast, bepaalt welke requirements prioriteit krijgen en is verantwoordelijk om de noden van de stakeholders te begrijpen en te vertegenwoordigen. Hierdoor is het dus vanzelfsprekend dat de persoon die deze rol toegewezen wordt, vertrouwd is met  tal van vaardigheden die eigen zijn aan business analyse. In deze rol blinkt de business analist dan ook uit, zoniet als product owner dan wel als iemand die de eigenlijke product owner steeds ondersteunt. Naast de functie van business advisor zou de business analist bovendien ook dienst kunnen doen als business coach, waarbij hij zijn skills en know-how kan inzetten als de analyse specialist van het team. Dit houdt onder andere in dat hij de samenwerking tussen het development team en de stakeholders vlot doet verlopen en voorbeelden genereert die gebruikt kunnen worden om duidelijk de wensen van de business te schetsen.\autocite{mcdonald2017does}

Binnen agile heeft ten slotte ook een evolutie plaatsgevonden naar multidisciplinaire teams die bestaan uit een optimale balans tussen technische en softskills.\autocite{heijnerelevante}

\subsection{De hypothese}

Doordat agile draait om werken met multidisciplinaire teams, bestaande uit leden die samen een optimale combinatie van soft and technische skills vormen, kunnen we besluiten dat de business analist een cruciale rol speelt binnen de agile methodologie. 


\section{interview}








\printbibliography[heading=bibintoc]

\end{document}