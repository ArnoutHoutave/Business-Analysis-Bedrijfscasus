%==============================================================================
% Sjabloon onderzoeksvoorstel bachproef
%==============================================================================
% Gebaseerd op document class `hogent-article'
% zie <https://github.com/HoGentTIN/latex-hogent-article>

% Voor een voorstel in het Engels: voeg de documentclass-optie [english] toe.
% Let op: kan enkel na toestemming van de bachelorproefcoördinator!
\documentclass{hogent-article}
\addbibresource{literatuur.bib}


\studyprogramme{Professionele bachelor toegepaste informatica}
\course{Bachelorproef}
\assignmenttype{Literatuurstudie bedrijfscasus}

\academicyear{2022-2023} 

% TODO: Werktitel
\title{De rol van de business-analist in agile projecten}

% TODO: Studentnaam en emailadres invullen
\author{Arnout Houtave}
\email{Arnout.Houtave@student.hogent.be}


% TODO: Medestudent
% Gaat het om een bachelorproef in samenwerking met een student in een andere
% opleiding? Geef dan de naam en emailadres hier
% \author{Yasmine Alaoui (naam opleiding)}
% \email{yasmine.alaoui@student.hogent.be}
\author{Dylan Pylser}
\email{dylan.pylyser@student.hogent.be}
\author{Lukas Van Dorpe}
\email{lukas.vandorpe@student.hogent.be}
\author{Laurens Arnauts}
\email{laurens.arnauts@student.hogent.be}


\specialisation{Functional \& Business Analysis}
\keywords{agile projecten, business-analist, bedrijfscase}

\begin{document}

\begin{abstract}
 Still to do.
\end{abstract}

\tableofcontents


\section{Opdrachtomschrijving}

Het agile manifesto stelt "Working software over comprehensive documentation". Betekent dit dat lastenboeken overbodig zijn? Wat is dan nog de taak van de business-analist? Is business-analyse in agile projecten nog altijd een rol of enkel een skill? Wordt in zo'n project de rol van de opdrachtgever ("de business") minder relevant of juist belangrijker?  

Het zijn dan ook over deze vragen dat wij als studenten een mening dienen vormen aan de hand van kritisch onderzoek. Dit onderzoek bestaat uit twee opeenvolgende luiken: Allereerst dienen wij een literatuurstudie uit te voeren, die we gebruiken om onze eigen hypthese op bovenstaande vragen vorm te geven. Onze hypothese in het achterhoofd houdend, zullen wij deze vervolgens toetsen bij ten minste twee bedrijven die al reeds met ons onderwerp in het werkveld geconfronteerd werden. Eens dat wij alle informatie verzameld hebben, dienen wij deze daarna af te toetsen tegen onze eigen hypothese en te kijken waardat deze overeenkomt en verschilt.


\section{Literatuurstudie}

\subsection{Inleiding}

Het agile manifesto stelt:

\begin{itemize}
  \item Mensen en hun onderlinge interactie boven processen en hulpmiddelen
  \item Werkende software boven allesomvattende documentatie
  \item Samenwerking met de klant boven contractonderhandelingen
  \item Inspelen op verandering boven het volgen van een plan
\end{itemize}
\autocite{fowler2001agile}

Aangezien er nergens in een agile omgeving, of in het agile manifesto, expliciet de taken van een business analist beschreven staan, zou men aanvankelijk kunnen stellen dat de rol van business analist mag geschrapt worden binnen het project.

Het betekent echter niet omdat de rol de business analist weinig of niet ter sprake komt wanneer men het over agile heeft, dat er niet gedaan wordt aan business analyse tijdens agile projecten. Er wordt immers verwacht dat, dankzij agile's focus op het voorzien van waarde voor de klant, ieder lid van het team zich op regelmatige basis en collaboratief zich bezig houdt met business analyse.Hiernaast dient er ook aandacht besteed te worden aan een bijkomende verandering dat agile werken teweeg brengt: Hoewel de rollen in agile veelal een algemenere vorm aannemen dan wanneer deze gebruikt worden in een tradtionelere omgeving, kan men nog steeds spreken over rollen.Één van deze rollen, de \emph{product owner}, vormt de ultieme beslisser en vertegenwoordiger wanneer het gaat over de noden van de business tijdens het project. Deze persoon legt onder andere de product visie vast, bepaalt welke requirements prioriteit hebben en is verantwoordelijk voor de noden van de stakeholders te begrijpen en te vertegenwoordigen. Hierdoor is het dus vanzelfsprekend dat de persoon die deze rol toegewezen wordt, vertrouwd is met een tal van vaardigheden die eigen zijn aan business analyse. In deze rol blinkt de business analist dan ook uit, zoniet als product owner dan wel als iemand dat de eigenlijke product owner steeds ondersteunt. Naast de functie van business advisor zou deze bovendien dienst doen als business coach, waarbij hij zijn skills en know-how kan inzetten als de analyse specialist. Dit houdt onder andere in in dat hij de samenwerking tussen het development team en de stakeholder vlot doet verlopen en voorbeelden genereerd die gebruikt kunnen worden om duidelijk de wensen van de business te schetsen.\autocite{mcdonald2017does}

Binnen agile heeft ten slotte ook een evolutie plaatsgevonden naar multidisciplinaire teams die bestaan uit een optimale balans van technische en softskills. Skills die uiteraard niet allemaal kunnen toebehoren aan één lid.\autocite{heijnerelevante}

\subsection{De hypothese}

Doordat agile draait om werken met multidisciplinaire teams, bestaande uit leden die samen een optimale combinatie van soft and technische skills vormen, kan men besluiten dat de business analist een cruciale rol speelt binnen de agile metholody. 


\section{interview}








\printbibliography[heading=bibintoc]

\end{document}